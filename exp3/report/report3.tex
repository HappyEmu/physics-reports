\documentclass{scrreprt}
\usepackage[english]{babel}
\usepackage[T1]{fontenc}
\usepackage{lmodern}
\usepackage{blindtext}
\usepackage[utf8]{inputenc}
\usepackage{siunitx} %For unit handling%
\renewcommand{\familydefault}{\sfdefault}
\newcommand{\unit}[1]{\ensuremath{\, \mathrm{#1}}}
\usepackage{amssymb, amsmath, cancel, ulem, graphicx, float, tabularx, multirow, bm}
\usepackage{amsmath}

\setcounter{secnumdepth}{5}
\setcounter{tocdepth}{5}

\renewcommand{\emph}[1]{\textit{#1}}

\author{Urs Gerber\\09-921-156 \and Gian-Luca Mateo\\11-113-545}
\date{14th of March 2013}

\title{Collision Experiment}
\subtitle{Practical course report}

\begin{document}

\maketitle

\tableofcontents
\newpage

\chapter{Experiment: Momentum Conservation}
\section{Introduction}
A collision is a short-time interaction between two or more objects. In essence there are two types of collisions: the \emph{elastic} collision and the \emph{inelastic} collision.\\
In an ideal \emph{elastic} collision no energy is lost during the course of the interaction, meaning no energy is lost due to heat or deformation. In this type of collision both energy and momentum are conserved.\\
In the inelastic case both bodies will stick to each other after the collision and travel at a common terminal velocity. Only momentum is conserved as energy is lost to deformation.
 
\subsection{Goal of the experiment}

\subsection{Theory}
\subsubsection{Preliminary exercises}
Using $A:=\frac{m_2}{m_1}$

\paragraph*{Task 1}
\begin{equation}
m_1v_1=m_1v_1'+m_2v_2'
\end{equation}
\begin{equation}
\Rightarrow v_2'=\frac{m_1}{m_2}(v_1-v_1')
\end{equation}
\begin{equation}
\frac{1}{2}m_1v_1^2=\frac{1}{2}m_1v_1'^2+\frac{1}{2}m_2v_2'^2
\end{equation}
\begin{equation}
\Rightarrow m_1v_1^2=m_1v_1'^2+\frac{m_1^2}{m_2}(v_1-v_1')^2
\end{equation}
\begin{equation}
\Rightarrow v_1^2=v_1'^2+\frac{m_1}{m_2}(v_1-v_1')^2
\end{equation}
\begin{equation}
\Rightarrow v_1^2-v_1^2=\frac{m_1}{m_2}(v_1-v_1')^2
\end{equation}
\begin{equation}
\Rightarrow (v_1+v_1')=\frac{m_1}{m_2}(v_1-v_1')
\end{equation}
\begin{equation}
\Rightarrow v_1+v_1'=\frac{1}{A}(v_1-v_1')
\end{equation}
\begin{equation}
\Rightarrow v_1-\frac{v_1}{A}=-v_1-\frac{v_1'}{A}
\end{equation}
\begin{equation}
\Rightarrow v_1\left(1-\frac{1}{A}\right)=-v_1'\left(1+\frac{1}{A}\right)
\end{equation}
\begin{equation}
\Rightarrow \frac{v_1'}{v_1}=-\frac{1-\frac{1}{A}}{1+\frac{1}{A}} = \frac{A-1}{-A-1}
\end{equation}
\begin{equation}
\Rightarrow \frac{T_1'}{T_1} = \left(\frac{A-1}{-(A+1)}\right)^2=\left(\frac{A-1}{A+1}\right)^2
\end{equation}

\paragraph*{Task 2}
\begin{equation}
\frac{T_{1,2}'}{T_1}=\frac{(m_1+m_2)v_1'^2}{m_1v_1^2}
\end{equation}
\begin{equation}
v_1'=v_1\frac{m_1}{m_1+m_2}
\end{equation}
\begin{equation}
\Rightarrow \frac{T_1'}{T_1}= \frac{(m_1+m_2)}{m_1}\frac{v_1^2\left(\frac{m_1}{m_1+m_2}\right)^2}{v_1^2}=\frac{m_1}{m_1+m_2}=\frac{1}{1+A}
\end{equation}
\begin{equation}
\Rightarrow \frac{Q}{T_1}=1-\frac{T_1'}{T_1}=1-\frac{1}{1+A}=\frac{A}{1+A}
\end{equation}
Q ist der Energieanteil von $T_1$, der dem System beim Stoss verloren geht. Thermische Energie, Deformationsenergie und Schall tragen zu Q bei.

\paragraph*{Task 3}
\begin{equation}
l^2=a^2+(l-h)^2=a^2+l^2-2lh+\underbrace{h^2}_{\approx 0}
\end{equation}
\begin{equation}
\Rightarrow h=\frac{a^2}{2l}
\end{equation}
\begin{equation}
\frac{1}{2}m_2v_2'^2=m_2gh_2 \mbox{ und } \frac{1}{2}m_1v_1^2=m_1gh_1
\end{equation}
\begin{equation}
\frac{T_2'}{T_1}=\frac{m_2gh_2}{m_1gh_1}=\frac{m_2\frac{a_2^2}{2l}}{m_1\frac{a_1^2}{2l}}=\frac{m_2a_2^2}{m_1a_1^2}=A\frac{a_2^2}{a_1^2}
\end{equation}

\paragraph*{Task 4}
\begin{figure}[H]
	\centering
  \includegraphics[width=0.9\textwidth]{diag/collision.pdf}
	\caption{$\frac{T_2'}{T_1}$ in dependance of $A$}
	\label{fig:task4}
\end{figure}
\noindent Figure \ref{fig:task4} shows the progression of $\frac{T_2'}{T_1}$ dependance of $A$ for an elastic and inelastic collision.\\
One can easily see that the maximum transfer of energy occurs at $A=1\Leftrightarrow m_1=m_2$

\paragraph*{Task 5}
We need the following formulae to analyze our data:
\begin{center}
\begin{tabular}{lrl}
Theory, elastic & $\frac{T_2'}{T_1} = $ &  $\frac{4A}{(A+1)^2}$ \\
Theory, inelastic &$\frac{T_2'}{T_1} =$ &  $\frac{A}{(A+1)^2}$ \\
Measurment & $\frac{T_2'}{T_1} = $ & $A \frac{a_2^2}{a_1^2}$ \\
Theory & $\frac{Q}{T_1} = $ & $\frac{A}{1+A}$ \\
Measurement & $\frac{Q}{T_1} =$ & $1-(1+A)\frac{a_2^2}{a_1^2}$ \\
\end{tabular}
\end{center}

\section{Experiment setup and execution}

\subsection{Used materials}
\begin{figure}[H]
	\centering
  \includegraphics[width=0.9\textwidth]{img/topview.jpg}
	\caption{All used materials}
	\label{fig:materials}
\end{figure}

\subsection{Assembly}
\begin{figure}[H]
	\centering
  \includegraphics[width=0.9\textwidth]{img/assembly.jpg}
	\caption{Final experiment assembly}
	\label{fig:assembly}
\end{figure}

\section{Measurements}
\subsection{Elastic Collision}
\begin{table}[H]
	\centering
	\begin{tabular}{ccp{1.5cm}ccp{1.5cm}rl}
 P: B &   T: A &            &  P: A &   T: B &            &  P: A &   T: C \\\cline{1-2}\cline{4-5}\cline{7-8}

     40.75 &         59 &            &      34.25 &         22 &            &         98 &         36 \\

     40.75 &         58 &            &      34.25 &         22 &            &         98 &         36 \\

     40.75 &         60 &            &      34.25 &         21 &            &         98 &         35 \\

     40.75 &         59 &            &      34.25 &         21 &            &         98 &         36 \\

     40.75 &         58 &            &      34.25 &         22 &            &         98 &         35 \\

           &            &            &            &            &            &            &            \\

 P: C &   T: A &            &  P: C &   T: B &            &  P: B &   T: C \\\cline{1-2}\cline{4-5}\cline{7-8}

     40.75 &         68 &            &      34.25 &         54 &            &         28 &         20 \\

     40.75 &         68 &            &      34.25 &         54 &            &         28 &         20 \\

     40.75 &         68 &            &      34.25 &         54 &            &         28 &         20 \\

     40.75 &         68 &            &      34.25 &         53 &            &         28 &         20 \\

     40.75 &         68 &            &      34.25 &         54 &            &         28 &         20 \\

           &            &            &            &            &            &            &            \\

 P: B &   T: B &            &            &            &            &            &            \\\cline{1-2}

     54.25 &         55 &            &            &            &            &            &            \\

     54.25 &         55 &            &            &            &            &            &            \\

     54.25 &         55 &            &            &            &            &            &            \\

     54.25 &         55 &            &            &            &            &            &            \\

     54.25 &         55 &            &            &            &            &            &            \\
\end{tabular}  
\caption{Gemessene Auslenkungen in mm für den elastischen Stoss. Die rechte Kugel traf auf die linke Kugel.}  
\end{table}
\subsection{Inelastic Collision}

\section{Analysis and Discussion}


\begin{thebibliography}{9}

\bibitem{physcript13}
  Peter Wurz,
  \emph{Anleitung zum Physikpraktikum}
  FS2013

\end{thebibliography}

\end{document}
