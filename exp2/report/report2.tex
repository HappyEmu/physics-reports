\documentclass[abstract=on]{scrreprt}
\usepackage[english]{babel}
\usepackage[T1]{fontenc}
\usepackage{lmodern}
\usepackage{blindtext}
\usepackage[utf8]{inputenc}
\usepackage{siunitx} %For unit handling%
\renewcommand{\familydefault}{\sfdefault}
\newcommand{\unit}[1]{\ensuremath{\, \mathrm{#1}}}
\usepackage{amssymb, amsmath, cancel, ulem, graphicx, float, tabularx, multirow, bm}
\usepackage{amsmath}

\setcounter{secnumdepth}{5}
\setcounter{tocdepth}{5}

\author{Urs Gerber\\09-921-156 \and Gian-Luca Mateo\\11-113-545}
\date{7th of March 2013}

\title{Transverse oscillation of a string}
\subtitle{Practical course report}

\begin{document}

\maketitle

\tableofcontents
\newpage

\chapter{Experiment: Transverse oscillation of a string}
\section{Introduction}

\subsection{Goal of the experiment}
\subsection{Theory}
\subsection{Error calculation}

\section{Experiment setup and execution}

\section{Measurements and Analysis}

\subsection[Frequency]{Frequency $\bm{f}$}

\begin{table}[H]
	\center
	\begin{tabular}{|c|c|cccccc|}
	\cline{2-8}
	 \multicolumn{1}{c|}{}& $n=$ & 1 & 2 & 3 & 4 & 5 & 6\\  \hline
	\multirow{2}{*}{$l=1\unit{m}$} & $F_1=14.7\unit{N}$ & 63.8 & 125.1 & 186.9 & 248.4 & 311.0 & 371.6\\
	& $F_2=29.4\unit{N}$ & 87.23 & 172.0 & 258.5 & 344.0 & 430.4 & 514.6\\
	\hline
	\end{tabular}
	\caption{String frequency in Hertz [Hz] at highest oscillation amplitude under distinct tensile forces $F_1, F_2$ and in oscillation modes $n=1,...,6$ with string length $l=1\unit{m}$}
\end{table}

\subsection[Mass per length]{Mass per length $\bm{\mu}$}

\subsection[Transervse propagation speed]{Transverse propagation speed $\bm{v}$}


\section{Discussion}

\begin{thebibliography}{9}

\bibitem{physcript13}
  Peter Wurz,
  \emph{Anleitung zum Physikpraktikum}
  FS2013

\end{thebibliography}

\end{document}
