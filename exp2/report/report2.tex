\documentclass[abstract=on]{scrreprt}
\usepackage[english]{babel}
\usepackage[T1]{fontenc}
\usepackage{lmodern}
\usepackage{blindtext}
\usepackage[utf8]{inputenc}
\usepackage{siunitx} %For unit handling%
\renewcommand{\familydefault}{\sfdefault}
\newcommand{\unit}[1]{\ensuremath{\, \mathrm{#1}}}
\usepackage{amssymb, amsmath, cancel, ulem, graphicx, float, tabularx, multirow, bm}
\usepackage{amsmath}

\setcounter{secnumdepth}{5}
\setcounter{tocdepth}{5}

\author{Urs Gerber\\09-921-156 \and Gian-Luca Mateo\\11-113-545}
\date{7th of March 2013}

\title{Transverse oscillation of a string}
\subtitle{Practical course report}

\begin{document}

\maketitle

\tableofcontents
\newpage

\chapter{Experiment: Transverse oscillation of a string}
\section{Introduction}

\subsection{Goal of the experiment}
\subsection{Theory}
\subsection{Error calculation}

\section{Experiment setup and execution}

\section{Measurements and Analysis}

\subsection[Frequency]{Frequency $\bm{f}$}

\begin{table}[H]
	\center
	\begin{tabular}{|c|c|cccccc|}
	\cline{2-8}
	 \multicolumn{1}{c|}{}& $n=$ & 1 & 2 & 3 & 4 & 5 & 6\\  \hline
	\multirow{2}{*}{$l=1\unit{m}$} & $K_1=14.7\unit{N}$ & 63.8 & 125.1 & 186.9 & 248.4 & 311.0 & 371.6\\
	& $K_2=29.4\unit{N}$ & 87.23 & 172.0 & 258.5 & 344.0 & 430.4 & 514.6\\
	\hline
	\end{tabular}
	\caption{String frequency $f$ in Hertz [Hz] at highest oscillation amplitude under distinct tensile forces $F_1, F_2$ and in oscillation modes $n=1,...,6$ with string length $l=1\unit{m}$}
\end{table}

\subsection[Mass per length]{Mass per length $\bm{\mu}$}

\begin{equation}
v=\sqrt{\frac{K}{\mu}}=\lambda_n f_n \Rightarrow \mu = \frac{K}{\lambda_n^2 f_n^2} 
\end{equation}

\begin{table}[H]
	\center
	\begin{tabular}{|c|c|cccccc|}
	\cline{2-8}
	 \multicolumn{1}{c|}{}& $n=$ & 1 & 2 & 3 & 4 & 5 & 6\\  \hline
	\multirow{2}{*}{$l=1\unit{m}$} & $K_1$ & 9.029e-4 & 9.393e-4 & 9.469e-4 & 9.530e-4 & 9.499e-4 & 9.581e-4\\
	& $K_2$ & 9.660e-4 & 9.938e-4 & 9.899e-4 & 9.938e-4 & 9.919e-4 & 9.992e-4\\
	\hline
	\end{tabular}
	\caption{Calculated mass per length $\mu$ [$\frac{\unit{kg}}{\unit{m}}$]}
\end{table}

\begin{table}[H]
\center
\begin{tabular}{|c|ccc|}
\cline{2-4}
\multicolumn{1}{c|}{}& $\bar{\mu}$ & $s_{\mu}$ & $s_{\bar{\mu}}$\\ \hline
$F_1$ & 9.417e-4 & 2.002e-5 & 8.954e-6 \\ \hline
$F_2$ & 9.891e-4 & 1.175e-5 & 5.255e-6\\ \hline
\end{tabular}
\caption{Mean $\bar{\mu}$, standard deviation $s_{\mu}$ and standard error of mean $s_{\bar{\mu}}$} of mass per length $\mu$ in [$\unit{\frac{kg}{m}}$]
\end{table}

\subsection[Transervse propagation speed]{Transverse propagation speed $\bm{v}$}

\begin{equation}
v=\lambda_n f_n
\end{equation}

\begin{table}[H]
	\center
	\begin{tabular}{|c|c|cccccc|}
	\cline{2-8}
	 \multicolumn{1}{c|}{}& $n=$ & 1 & 2 & 3 & 4 & 5 & 6\\  \hline
	\multirow{2}{*}{$l=1\unit{m}$} & $K_1$ & 127.6 & 125.1 & 124.6 & 124.2 & 124.4 & 123.867\\
	& $K_2$ & 174.46 & 172.0 & 172.333 & 172.0 & 172.16 & 171.533\\
	\hline
	\end{tabular}
	\caption{Calculated transverse propagation speed $v$ in [$\unit{\frac{m}{s}}]$}
\end{table}

\begin{table}[H]
\center
\begin{tabular}{|c|ccc|}
\cline{2-4}
\multicolumn{1}{c|}{}& $\bar{v}$ & $s_{v}$ & $s_{\bar{v}}$\\ \hline
$K_1$ & 124.961 & 1.357 & 0.607 \\ \hline
$K_2$ & 172.414 & 1.037 & 0.464\\ \hline
\end{tabular}
\caption{Mean $\bar{\mu}$, standard deviation $s_{\mu}$ and standard error of mean $s_{\bar{\mu}}$} of mass per length $\mu$ in [$\unit{\frac{kg}{m}}$]
\end{table}

\section{Discussion}

\begin{thebibliography}{9}

\bibitem{physcript13}
  Peter Wurz,
  \emph{Anleitung zum Physikpraktikum}
  FS2013

\end{thebibliography}

\end{document}
