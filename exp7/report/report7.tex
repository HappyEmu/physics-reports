\documentclass{scrreprt}
\usepackage[english]{babel}
\usepackage[T1]{fontenc}
\usepackage{lmodern}
\usepackage{blindtext}
\usepackage[utf8]{inputenc}
\usepackage{siunitx} %For unit handling%
\renewcommand{\familydefault}{\sfdefault}
\newcommand{\unit}[1]{\ensuremath{\, \mathrm{#1}}}
\usepackage{amssymb, amsmath, cancel, ulem, graphicx, float, tabularx, multirow, bm}
\usepackage{amsmath}
\usepackage{caption}
\usepackage{subcaption}
\usepackage{mathtools}
\usepackage{tikz}
\newcommand*\circled[1]{\tikz[baseline=(char.base)]{
            \node[shape=circle,draw,inner sep=1pt] (char) {#1};}}
\renewcommand{\phi}{\varphi}

\setcounter{secnumdepth}{5}
\setcounter{tocdepth}{5}

\author{Urs Gerber\\09-921-156 \and Gian-Luca Mateo\\11-113-545}
\date{18th of April 2013}

\title{Coupled pendulums}
\subtitle{Practical course report}

\begin{document}

\maketitle

\tableofcontents
\newpage

\chapter{Experiment: Coupled pendulums}

\section{Introduction}

\subsection{Goal of the experiment}
The goal of this experiment is to measure and analyse the characteristics of a coupled pendulum. For that we measure the cycle duration of the coupled pendulum in three different oscillation modes: in-phase, opposite in-phase and paraphase.

\subsection{Theory}

\subsubsection{Mathematical pendulum}
\paragraph{Equation of motion}
The equation of motion for the mathematical pendulum with thread length $l$ is given by
\begin{equation}
\ddot{\phi} + \frac{g}{l} \cdot \sin{\phi} = 0 \xRightarrow{\sin{\phi}\approx\phi} \ddot{\phi} + \frac{g}{l} \cdot \phi = 0
\end{equation}
with solution 
\begin{equation}
\phi (t) = \hat{\phi} \cdot \sin{\left( \sqrt{\frac{g}{l}} \cdot t + \phi_0 \right)} 
\end{equation}
The oscillation period $T_0$ of the mathematical pendulum is
\begin{equation}
T_0 = 2 \pi \sqrt{\frac{l}{g	}}
\end{equation}

\subsubsection{Coupled physical pendulum}

\paragraph{Equation of motion}
The equation of motion for one physical pendulum is given by

\begin{equation}
J\cdot \ddot{\phi} = M
\end{equation}
where $J$ is the moment of inertia with respect to the axis of rotation $O$ and $M$ the repulsive angular moment.

\paragraph{Moment equations}
The moment equations for pendulum 1 and pendulum 2 are:
\begin{align}
M_1 &= \overbrace{-D_g\cdot \phi_1}^{\text{directional moment}} + \overbrace{D_f \cdot (\phi_1 - \phi_2)}^{\text{coupling moment}}  \\
M_2 &= -D_g\cdot \phi_2 - D_f \cdot (\phi_1 - \phi_2)
\end{align}

After inserting the moment equations into the equation of motion and solving the differential equations, we get the two solutions

\begin{align}
\phi_1(t) &= A \cdot \cos{(\omega t + \delta)} - B \cdot \cos{(\Omega t + \Delta)}\\
\phi_2(t) &= A \cdot \cos{(\omega t + \delta)} + B \cdot \cos{(\Omega t + \Delta)}
\end{align}
with natural frequencies $\omega$ and $\Omega$
\begin{align}
\omega &= \sqrt{\frac{D_g}{J}}\\
\Omega &= \sqrt{\frac{D_g+2 D_f}{J}}\\
\end{align}
One can see that the overall motion of each pendulum is composed of a superposition of two harmonic oscillations with different frequencies (beat).

\paragraph{Initial values}
We distinguish the following three cases for pendulum 1 and pendulum 2. At $t=0$ the initial deflection of the pendulums may be
\begin{itemize}
\item in-phase: $\phi_1 = \Phi, \quad \phi_2 = \Phi$
\item opposite in-phase: $\phi_1 = -\Phi, \quad \phi_2 = +\Phi$
\item paraphase: $\phi_1 = 0, \quad \phi_2 = \Phi$
\end{itemize}

\subsection{Uncertainty analysis}

\section{Experiment setup and execution}

\subsection{Used materials}
The materials used in this experiment are the following:
\begin{itemize}
\item list
\end{itemize}

\subsection{Assembly}

\section{Measurements}

\section{Analysis and Discussion}

\section{Conclusion}

\begin{thebibliography}{9}

\bibitem{physcript13}
  Peter Wurz,
  \emph{Anleitung zum Physikpraktikum}
  FS2013

\end{thebibliography}

\end{document}
