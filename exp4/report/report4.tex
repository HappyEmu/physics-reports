\documentclass{scrreprt}
\usepackage[english]{babel}
\usepackage[T1]{fontenc}
\usepackage{lmodern}
\usepackage{blindtext}
\usepackage[utf8]{inputenc}
\usepackage{siunitx} %For unit handling%
\renewcommand{\familydefault}{\sfdefault}
\newcommand{\unit}[1]{\ensuremath{\, \mathrm{#1}}}
\usepackage{amssymb, amsmath, cancel, ulem, graphicx, float, tabularx, multirow, bm}
\usepackage{amsmath}

\setcounter{secnumdepth}{5}
\setcounter{tocdepth}{5}

\author{Urs Gerber\\09-921-156 \and Gian-Luca Mateo\\11-113-545}
\date{21th of March 2013}

\title{Airfoil in a wind channel}
\subtitle{Practical course report}

\begin{document}

\maketitle

\tableofcontents
\newpage

\chapter{Experiment: Airfoil in a wind channel}
\section{Introduction}
\subsection{Goal of the experiment}
The goal of this experiment is to determine the angle of attack where for a given airfoil the ratio of air drag to lift is optimal. Furthermore, the lift is to be calculated by measuring the pressure conditions on the surface of the airfoil.
\subsection{Theory}
\subsubsection{Preliminary exercises}
\section{Experiment setup and execution}

\subsection{Used materials}


\subsection{Assembly}

\section{Measurements and analysis}
\section{Discussion}

\begin{thebibliography}{9}

\bibitem{physcript13}
  Peter Wurz,
  \emph{Anleitung zum Physikpraktikum}
  FS2013

\end{thebibliography}

\end{document}
