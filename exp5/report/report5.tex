\documentclass{scrreprt}
\usepackage[english]{babel}
\usepackage[T1]{fontenc}
\usepackage{lmodern}
\usepackage{blindtext}
\usepackage[utf8]{inputenc}
\usepackage{siunitx} %For unit handling%
\renewcommand{\familydefault}{\sfdefault}
\newcommand{\unit}[1]{\ensuremath{\, \mathrm{#1}}}
\usepackage{amssymb, amsmath, cancel, ulem, graphicx, float, tabularx, multirow, bm}
\usepackage{amsmath}
\usepackage{caption}
\usepackage{subcaption}
\usepackage{tikz}
\newcommand*\circled[1]{\tikz[baseline=(char.base)]{
            \node[shape=circle,draw,inner sep=1pt] (char) {#1};}}
\renewcommand{\phi}{\varphi}

\setcounter{secnumdepth}{5}
\setcounter{tocdepth}{5}

\author{Urs Gerber\\09-921-156 \and Gian-Luca Mateo\\11-113-545}
\date{28th of March 2013}

\title{Forced Oscillation}
\subtitle{Practical course report}

\begin{document}

\maketitle

\tableofcontents
\newpage

\chapter{Experiment: Forced Oscillation}
\section{Introduction}
\subsection{Goal of the experiment}
The goal of this experiment is to determine the resonant frequency and damping coefficient of an osillating system using two different methods: using forced and free oscillations.

\subsection{Theory}
The equation of motion for the axial rotation of a rigid body equates to:
\begin{equation}
J\cdot \ddot{\phi} = \sum_i M_i
\end{equation}
where $\phi$ is the angle of revolution, $J$ the moment of inertia and $M_i$ the angular momenta in relation to the rotation axis.

\subsubsection{Free damped oscillation}
Equation of motion:
\begin{equation}
J\ddot{\phi} + \underbrace{R \dot{\phi}}_{\text{frictional moment}} + \underbrace{D \phi}_{\text{repulsive moment}} = 0
\end{equation}
where $R$ is the damping coefficent, and $D$ is the spring constant. Additionally, we define
\begin{equation}
\omega_0^2 \doteq \frac{D}{J}, \quad \alpha \doteq \frac{R}{2J}
\end{equation}

The solution in case of weak damping ($\alpha^2 - \omega_0^2 < 0$) is
\begin{equation}
\phi(t) = A e^{-\alpha t} \cos{\left(\omega t - \beta\right)} \label{eq:freedampedsolution}
\end{equation}
with
\begin{equation} 
\omega \doteq \sqrt{\omega_0^2-\alpha^2}
\label{omega0}
\end{equation}
where $A$ and $\beta$ are integration constants determined by initial conditions.

\subsubsection{Paranthesis: Preliminary Exercises}
By assumption: $\varphi(t) = (A_1e^{\omega t} + A_2e^{-\omega t})e^{-\alpha t}$\\
\paragraph*{Task} at time $t=0$ let $\varphi(0) = \varphi_0$ and $\dot{\varphi}(0) = 0$. Find $A_1$ and $A_2$.\\
\paragraph*{Solution}
\begin{equation}
\varphi_0 \stackrel{!}= \varphi(0) = A_1 + A_2
\end{equation}
\begin{equation}
\Rightarrow A_2 = \varphi_0-A_1 \mbox{ und } A_1 = \varphi_0-A_2
\end{equation}
\begin{equation}
\dot{\varphi}(t) = (\omega A_1 e^{\omega t}-\omega A_2 e^{-\omega t})e^{-\alpha t} + (A_1e^{\omega t}+A_2e^{-\omega t})(-\alpha)e^{-\alpha t}
\end{equation}
\begin{equation}
\Rightarrow \dot{\varphi}(0) = \omega (A_1-A_2) - \alpha(A_1+A_2) \stackrel{!} = 0
\end{equation}
\begin{equation}
\Rightarrow \omega(A_1-A_2) = \alpha(A_1+A_2) = \alpha \varphi_0
\end{equation}
\begin{equation}
\Rightarrow A_1-A_2 = \frac{\alpha \varphi_0}{\omega}
\end{equation}
\begin{equation}
\Rightarrow A_1 -(\varphi_0-A_1) = \frac{\alpha \varphi_0}{\omega}
\end{equation}
\begin{equation}
\Rightarrow A_1=\frac{\varphi_0}{2 \omega} (\omega + \alpha)
\end{equation}
\begin{equation}
\Rightarrow A_2=\frac{\varphi_0}{2 \omega} (\omega - \alpha)
\end{equation}

\subsubsection{Forced damped oscillation}
Equation of motion:
\begin{equation}
J\ddot{\phi} + \underbrace{R \dot{\phi}}_{\text{frictional moment}} + \underbrace{D \phi}_{\text{repulsive moment}}  = \underbrace{M_0 \sin{\Omega t}}_{\text{propulsive time dependant moment}}
\end{equation}

The solution in case of weak damping ($\alpha^2 - \omega_0^2 < 0$) is
\begin{equation}
\phi(t) = A \sin{(\Omega t - \delta)} + B e^{-\alpha t} \cos{(\omega t - \beta)} \label{eq:forceddampedsolution}
\end{equation}

The oscillation's amplitude $A$ is dependant on the amplitude of the stimulus $\Omega$. $A$ reaches its maximum value at $\Omega_R$:
\begin{equation}
\Omega_R = \sqrt{\omega_0^2-2\alpha^2}
\end{equation}

\section{Experiment setup and execution}
\subsection{Used materials}
The materials used in this experiment are the following:
\begin{itemize}
\item A hinged metal disc with an indicator at its top connected indirectly to an electric motor using a spring
\item A ring with a printed scale with greater inner radius than the metal disc's diameter
\item An eddy brake
\item An electric engine with 2 analogue regulating knobs for speed regulation
\item A power supply with two DC outputs, one adjustable at the front at 1-15V / 2A max and the other one at the back fixed at 25V / 0.2A
\item A mechanical stopwatch (0.2s steps)
\item A multimeter (Model FLUKE 73 Multimeter)
\item Six electric wires
\end{itemize}

\subsection{Assembly}

\begin{figure}[H]
	\centering
  \includegraphics[width=0.9\textwidth]{img/assembly.jpg}
	\caption{The experiment assembly}
	\label{fig:assembly}
\end{figure}

For our measurements, we connect the front output of the power supply to the multimeter and then to the eddy brake and set the multimeter to read the amperage of direct current. The electric engine is directly connected to the back output of the power supply and attached to the metal disc using a spring as shown in \ref{fig:assembly}.
\subsubsection{Forced oscillation}
For our first measurement series, we set the eddy brake to a certain current and then measured the maximal amplitude of the oscillation at different generator frequencies.
\subsubsection{Free oscillation}
For our second mesaurement series we locked the motor and set the eddy brake to the same currents as before plus some others. Then, we moved the disc to an initial deflection and released it, measuring the progression of the maximal amplitudes over time.

\section{Measurements}
\subsection{forced oscillation}
\begin{figure}[H]
	\centering
  \includegraphics[width=0.9\textwidth]{diag/measurements_amplitudes.pdf}
	\caption{Forced oscillation}
	\label{fig:forcedoscillation}
\end{figure}

\subsection{free oscillation}
\begin{figure}[H]
	\centering
  \includegraphics[width=0.9\textwidth]{diag/3A.pdf}
	\caption{Amplitude history over time and exponential approximation}
	\label{fig:0.3A}
\end{figure}

\begin{figure}[H]
	\centering
  \includegraphics[width=0.9\textwidth]{diag/5A.pdf}
	\caption{Amplitude history over time and exponential approximation}
	\label{fig:0.5A}
\end{figure}

\begin{figure}[H]
	\centering
  \includegraphics[width=0.9\textwidth]{diag/65A.pdf}
	\caption{Amplitude history over time and exponential approximation}
	\label{fig:0.65A}
\end{figure}

\begin{figure}[H]
	\centering
  \includegraphics[width=0.9\textwidth]{diag/8A.pdf}
	\caption{Amplitude history over time and exponential approximation}
	\label{fig:0.8A}
\end{figure}

TODO: insert Frequency measurements!
TODO: change AC to DC

\section{Analysis and Discussion}
We used two methods for determining the decay coefficient $\alpha$ and the angular frequency. The first method implies making an exponential fit to the exponential values of the amplitude history as seen in figures \ref{fig:0.3A} - \ref{fig:0.8A}.  By comparing the formula of the exponential fit to the formula !!!makeref!!! and seeing that $\lvert cos(\omega\cdot t-\beta)\rvert$ is $1$ $\forall \omega \cdot t = k \cdot \pi (k \in \mathbb{N})$, we deduce form of the exponential fit formula to be:
\begin{equation}
y = Ae^{- \alpha \cdot t}
\end{equation}
where $A$ is the initial deflection.
We can now read $\alpha$ directly from the fit formulae:

\begin{table}[H]
\center
\begin{tabular}{lcl}
$I = 0.3 \unit{A}$ &:& $\alpha = 0.079\unit{Hz}$\\
$I = 0.5 \unit{A}$ &:& $\alpha = 0.169\unit{Hz}$\\
$I = 0.65 \unit{A}$ &:& $\alpha = 0.251\unit{Hz}$\\
$I = 0.8 \unit{A}$ &:& $\alpha = 0.380\unit{Hz}$
\end{tabular}
\end{table}

If we now proceed to calculating the resonant angular frequency of the undampened system, we deduce

\begin{equation}
\omega_0 = \sqrt{\omega^2+\alpha^2}, \quad \omega = \frac{2 \pi}{T} \quad \Longrightarrow \quad \omega_0 = \sqrt{\frac{4\cdot\pi^2}{T^2} + \alpha^2}
\end{equation}
and get the following values:

\begin{table}[H]
\center
\begin{tabular}{lcl}
$I = 0.3 \unit{A}$ &:& $\omega_0 = 3.0810\unit{Hz}$\\
$I = 0.5 \unit{A}$ &:& $\omega_0 = 3.0846\unit{Hz}$\\
$I = 0.65 \unit{A}$ &:& $\omega_0 = 3.0902\unit{Hz}$\\
$I = 0.8 \unit{A}$ &:& $\omega_0 = 3.0649\unit{Hz}$
\end{tabular}
\end{table}

Which yields:
\begin{equation}
\omega_0 = 3.0802 \pm 0.005432 \unit{Hz}
\end{equation}

The second method uses the measurements depicted in \ref{fig:forcedoscillation}.
We first determine $\alpha$ by means of graphical analysis:

\begin{figure}[H]
	\centering
  \includegraphics[width=0.9\textwidth]{diag/Width_measurement.png}
	\caption{Graphical analysis for determining the $\alpha$ value}
	\label{fig:graphical}
\end{figure}

As stated in \cite{physcript13}, page 80, $\alpha$ is defined as half the width of the resonance curve at $y=\frac{A^*_{max}}{\sqrt{2}}$. This yields the following values:

\begin{table}[H]
\center
\begin{tabular}{lcl}
$I = 0.3 \unit{A}$ &:& $\alpha = 0.0472\unit{Hz}$\\
$I = 0.8 \unit{A}$ &:& $\alpha = 0.3491\unit{Hz}$
\end{tabular}
\end{table}

From the same diagram, we can then read the maximal resonance frequencies $\omega$

\begin{table}[H]
\center
\begin{tabular}{lcl}
$I = 0.3 \unit{A}$ &:& $\omega = 3.1104\unit{Hz}$\\
$I = 0.8 \unit{A}$ &:& $\omega = 3.0218\unit{Hz}$
\end{tabular}
\end{table}

Using formula \ref{omega0} and the values for alpha obtained from the graphical analysis we can again calculate $\omega_0$:
\begin{table}[H]
\center
\begin{tabular}{lcl}
$I = 0.3 \unit{A}$ &:& $\omega_0 = 3.1108\unit{Hz}$\\
$I = 0.8 \unit{A}$ &:& $\omega_0 = 3.0419
\unit{Hz}$
\end{tabular}
\end{table}

which yields $\omega_0 = 3.07635 \pm 0.02435 \unit{Hz}$
\subsection{Final Comparison of Results}

\begin{table}[H]
\center
\begin{tabular}{lrrr}
&Free oscillation& Forced oscillation& Relative error\\
$\alpha$(0.3 A)& 0.0790 Hz & 0.0472 Hz & -40.25\%\\
$\alpha$(0.8 A)& 0.3800 Hz & 0.3491 Hz & -8.13\%\\
$\omega_0$&$3.0802 \pm 0.005432 \unit{Hz}$&$ 3.07635 \pm 0.02435 \unit{Hz}$ & -0.12\%\\
\end{tabular}
\end{table}

The results of both methods are quite close to each other, apart from the decay coefficient at 0.3 A on the eddy brake. We aggregate that to the problems we faced in the graphical analysis (i.e. the curve being very steep and having very different results for very similar measurements) and assume that the measurement of the decay coefficient was much more precise for the free oscillation, because we used regression analysis, which tends to yield good results. 

\section{Conclusion}

Taking everything stated in this report into consideration, we think our results are good and precise, as both methods yield very similar results for the resonant frequency of the free undamped system. As there are no theoretical values for our experiment setup we can use, we have no further method of evaluation.


\begin{thebibliography}{9}

\bibitem{physcript13}
  Peter Wurz,
  \emph{Anleitung zum Physikpraktikum}
  FS2013

\end{thebibliography}

\end{document}
