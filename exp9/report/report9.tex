\documentclass{scrreprt}
\usepackage[english]{babel}
\usepackage[T1]{fontenc}
\usepackage{lmodern}
\usepackage{blindtext}
\usepackage[utf8]{inputenc}
\usepackage{siunitx} %For unit handling%
\renewcommand{\familydefault}{\sfdefault}
\newcommand{\unit}[1]{\ensuremath{\, \mathrm{#1}}}
\usepackage{amssymb, amsmath, cancel, ulem, graphicx, float, tabularx, multirow, bm}
\usepackage{amsmath}
\usepackage{caption}
\usepackage{subcaption}
\usepackage{mathtools}
\usepackage{tikz}
\usepackage{commath}
\newcommand*\circled[1]{\tikz[baseline=(char.base)]{
            \node[shape=circle,draw,inner sep=1pt] (char) {#1};}}
\renewcommand{\phi}{\varphi}


\setcounter{secnumdepth}{5}
\setcounter{tocdepth}{5}

\author{Urs Gerber\\09-921-156 \and Gian-Luca Mateo\\11-113-545}
\date{2nd of May 2013}

\title{Spinning Top}
\subtitle{Practical course report}

\begin{document}

\maketitle

\tableofcontents
\newpage

\chapter{Experiment: Spinning Top}

\section{Introduction}

\subsection{Goal of the experiment}
The goal of this experiment is to analize and measure the properties of a spinning top. More precisely, we are first and foremost interested in the top's precession period when applying variable momenta at different angular frequencies.
 
\subsection{Theory}
A top consists of a rigid body which is free to rotate around a fixed point $0$. Generally said, the motion is a rotation around a momentary rotation axis with an arbitrary orientation.\\
In this experiment, however, we use a much simpler variation of the general top, called a rotationally symmetric top, which makes the equations of motion much simpler.

\subsubsection{Moment of Inertia}

\begin{equation}
\label{eq:inertia}
J = \frac{2mgh}{\omega_0^2} - m R^2
\end{equation}

\subsubsection{Precession period $T_P$}

\begin{equation}
\Omega_P = \frac{mg\cdot l}{J\cdot \omega}
\end{equation}

\begin{equation}
T_P = \frac{2\pi}{\Omega_P} = \frac{2\pi J \omega_0}{mgl}
\end{equation}

\subsubsection{Parenthesis: Preliminary Exercises}

\subsubsection{Error analysis}
\paragraph*{Moment of Inertia}

\begin{equation}
s_J^2 = 4 m^2 \left( r^2 s_r^2 + \frac{4 g^2 h^2 s_w^2}{\omega^6} \right)
\end{equation}

\section{Experiment setup and execution}

\subsection{Used materials}
The materials used in this experiment are the following:
\begin{itemize}
\item A rotationally symmetric top, hinged and turnable in the z-axis
\item An electric engine, attached to the top
\item A piece of string, about $1 \unit{m}$ in length, attachable to the metal disk of the top, with a hook to attach weights
\item Four weights, $0.2 \unit{kg}$, $0.46 \unit{kg}$, $1 \unit{kg}$, $2 \unit{kg}$ respectively
\item A stroboscope with adjustable frequency
\end{itemize}

\subsection{Assembly and Execution}

\section{Experiment Results}

\section{Analysis and Discussion}

\subsection{Moment of inertia $J$}
To calculate the moment of inertia $J$ of the spinning disk, we use formula \ref{eq:inertia} and our measured values for $\omega_0$:

\begin{table}[H]
\centering
\begin{tabular}{|l|cccc|}
\hline
$h = 1\unit{m}$ & $ m = 0.2 \unit{kg}$ & $m = 0.46 \unit{kg}$ & $m = 1 \unit{kg}$ & $m = 2 \unit{kg}$\\
\hline\hline
$\omega_0$ [$\unit{rad/s}$] & $9.075 \pm 0.0631$ & $12.758 \pm 0.0374$ & $17.190 \pm 0.432$ & $21.546 \pm 0.159$\\ \hline
$J$ [$\unit{kg\cdot m^2}$] & $0.0440 \pm 0.00331$  & $0.0471 \pm 0.00072$ & $0.0482 \pm 0.0033$ & $0.0481 \pm 0.00334$ \\ \hline
\end{tabular}
\label{tab:inertia_results}
\caption{The resulting momenta of inertia $J$ for different values of $m$}
\end{table}

To calculate the effective moment of inertia $J$, we average all the momenta of inertia above, which yields

\begin{equation}
J = (0.0468	 \pm 0.000974) \unit{kg\cdot m^2}
\end{equation}

\subsection{Precession Period $T_P$}
\subsection{Sources of error}

\section{Conclusion}

\begin{thebibliography}{9}

\bibitem{physcript13}
  Peter Wurz,
  \emph{Anleitung zum Physikpraktikum}
  FS2013

\end{thebibliography}

\end{document}
