\documentclass{scrreprt}
\usepackage[english]{babel}
\usepackage[T1]{fontenc}
\usepackage{lmodern}
\usepackage{blindtext}
\usepackage[utf8]{inputenc}
\usepackage{siunitx} %For unit handling%
\renewcommand{\familydefault}{\sfdefault}
\newcommand{\unit}[1]{\ensuremath{\, \mathrm{#1}}}
\usepackage{amssymb, amsmath, cancel, ulem, graphicx, float, tabularx, multirow, bm}
\usepackage{amsmath}
\usepackage{caption}
\usepackage{subcaption}
\usepackage{mathtools}
\usepackage{tikz}
\usepackage{commath}
\newcommand*\circled[1]{\tikz[baseline=(char.base)]{
            \node[shape=circle,draw,inner sep=1pt] (char) {#1};}}
\renewcommand{\phi}{\varphi}


\setcounter{secnumdepth}{5}
\setcounter{tocdepth}{5}

\author{Urs Gerber\\09-921-156 \and Gian-Luca Mateo\\11-113-545}
\date{2nd of May 2013}

\title{Gyroscope}
\subtitle{Practical course report}

\begin{document}

\maketitle

\tableofcontents
\newpage

\chapter{Experiment: Gyroscope}

\section{Introduction}

\subsection{Goal of the experiment}
The goal of this experiment is to analize and measure the properties of a spinning gyroscope. More precisely, we are first and foremost interested in the gyroscope's precession period when applying variable momenta at different angular frequencies.
 
\subsection{Theory}
\subsection{Error analysis}

\section{Experiment setup and execution}

\subsection{Used materials}
The materials used in this experiment are the following:
\begin{itemize}
\item ...
\end{itemize}

\subsection{Assembly and Execution}

\section{Analysis and Discussion}

\subsection{Sources of error}

\section{Conclusion}

\begin{thebibliography}{9}

\bibitem{physcript13}
  Peter Wurz,
  \emph{Anleitung zum Physikpraktikum}
  FS2013

\end{thebibliography}

\end{document}
