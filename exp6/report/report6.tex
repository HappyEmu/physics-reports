\documentclass{scrreprt}
\usepackage[english]{babel}
\usepackage[T1]{fontenc}
\usepackage{lmodern}
\usepackage{blindtext}
\usepackage[utf8]{inputenc}
\usepackage{siunitx} %For unit handling%
\renewcommand{\familydefault}{\sfdefault}
\newcommand{\unit}[1]{\ensuremath{\, \mathrm{#1}}}
\usepackage{amssymb, amsmath, cancel, ulem, graphicx, float, tabularx, multirow, bm}
\usepackage{amsmath}
\usepackage{caption}
\usepackage{subcaption}
\usepackage{tikz}
\newcommand*\circled[1]{\tikz[baseline=(char.base)]{
            \node[shape=circle,draw,inner sep=1pt] (char) {#1};}}
\renewcommand{\phi}{\varphi}

\setcounter{secnumdepth}{5}
\setcounter{tocdepth}{5}

\author{Urs Gerber\\09-921-156 \and Gian-Luca Mateo\\11-113-545}
\date{11th of March 2013}

\title{Lenses}
\subtitle{Practical course report}

\begin{document}

\maketitle

\tableofcontents
\newpage

\chapter{Experiment: Lenses}

\section{Introduction}

\subsection{Goal of the experiment}
The goal of this experiment is to determine the focal distance of a set of lenses using the method proposed by German scientist Friedrich Wilhelm Bessel.

\subsection{Theory}
\subsubsection{Lens equation}
The commonly known lens formula is
\begin{equation}
\frac{1}{f} = \frac{1}{g} + \frac{1}{b}
\end{equation}
where $f$ is the focal distance, $g$ is the object distance and $b$ is the image distance.

\subsubsection{Lense pair}
The lens equation for a pair of lenses $L_1$ and $L_2$ looks a bit different:
\begin{equation}
\frac{1}{f}= \frac{1}{f_1} + \frac{1}{f_2} - \frac{d}{f_1 \cdot f_2}
\end{equation}
where $f_i$ is the focal distance of lense $L_i$ and $d$ is the distance between $L_1$ and $L_2$

\subsubsection{Bessel's method}
We use the method proposed by Bessel to find the focal distance $f$ of a lense by experimental means.

\section{Experiment setup and execution}

\subsection{Used materials}
The materials used in this experiment are the following:
\begin{itemize}
\item a projector which projects a static image
\item a milk glass screen
\item a rail with imprinted scale
\item 4 different lenses, labelled 1C to 4C
\item Some mounts for attaching all of the above to the rail
\end{itemize}
\subsection{Assembly}
For our measurements, the screen is attached at one end of the rail and the projector at the other, facing the screen. In between, a lense is placed and its position varied until two points are found where a sharp image appears on the screen. If we found no positions, we attached a second lens to the first lens and repeated the measurement. This whole process is repeated 5 times for each lens.

\section{Measurements}

\section{Analysis and Discussion}

\section{Conclusion}

\begin{thebibliography}{9}

\bibitem{physcript13}
  Peter Wurz,
  \emph{Anleitung zum Physikpraktikum}
  FS2013

\end{thebibliography}

\end{document}
