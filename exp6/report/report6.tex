\documentclass{scrreprt}
\usepackage[english]{babel}
\usepackage[T1]{fontenc}
\usepackage{lmodern}
\usepackage{blindtext}
\usepackage[utf8]{inputenc}
\usepackage{siunitx} %For unit handling%
\renewcommand{\familydefault}{\sfdefault}
\newcommand{\unit}[1]{\ensuremath{\, \mathrm{#1}}}
\usepackage{amssymb, amsmath, cancel, ulem, graphicx, float, tabularx, multirow, bm}
\usepackage{amsmath}
\usepackage{caption}
\usepackage{subcaption}
\usepackage{tikz}
\newcommand*\circled[1]{\tikz[baseline=(char.base)]{
            \node[shape=circle,draw,inner sep=1pt] (char) {#1};}}
\renewcommand{\phi}{\varphi}

\setcounter{secnumdepth}{5}
\setcounter{tocdepth}{5}

\author{Urs Gerber\\09-921-156 \and Gian-Luca Mateo\\11-113-545}
\date{11th of March 2013}

\title{Lenses}
\subtitle{Practical course report}

\begin{document}

\maketitle

\tableofcontents
\newpage

\chapter{Experiment: Lenses}

\section{Introduction}

\subsection{Goal of the experiment}
The goal of this experiment is to determine the focal distance of a set of lenses using the method proposed by German scientist Friedrich Wilhelm Bessel.

\subsection{Theory}
\subsubsection{preliminary exercises}

\textbf{Task 1}
\begin{equation}
	f=18cm\quad G=46cm\quad B=1cm
\end{equation}
\begin{equation}
	b=f\frac{B+G}{G}=18.391cm
\end{equation}
\begin{equation}
	g=f\frac{B+G}{B}=846cm
\end{equation}
\\
\textbf{Task 2}
\begin{equation}
	f=5cm\quad b=6cm
\end{equation}
\begin{equation}
	g=\frac{bf}{b-f}=30cm
\end{equation}
\\
\textbf{Task 3}
\begin{equation}
	G=30cm\quad g=500cm\quad B=1cm
\end{equation}
\begin{equation}
	f=\frac{gB}{B+G}=16.129cm
\end{equation}
\textbf{Task 4}
\begin{equation}
\sum n_i s_i \stackrel{!}= \mbox{Minimum}
\end{equation}
\begin{equation}
\Rightarrow n_1 \sqrt{(a_1+x)^2+d_1^2}+n_1\sqrt{(a_2-x)^2+d_2^2} \stackrel{!}= \mbox{Minimum}
\end{equation}
\begin{equation}
\Rightarrow \frac{d}{dx} \left(\sqrt{(a_1+x)^2+d_1^2}+\sqrt{(a_2+x)^2+d_2^2}\right) \stackrel{!}= 0
\end{equation}
\begin{equation}
\frac{1}{2} \frac{2(a_1+x)}{\sqrt{(a_1+x)^2+d_1^2}}+\frac{1}{2} \frac{-2(a_2-x)}{\sqrt{(a_2-x)^2+d_2^2}} \stackrel{!}= 0
\end{equation}
\begin{equation}
\frac{(a_1+x)^2}{(a_1+x)^2+d_1^2} = \frac{(a_2-x)^2}{(a_2-x)^2+d_2^2}
\end{equation}
This obviously holds for x=0 $\Rightarrow$ x=0 satisfies the minimum condition.
\\ \\
law of refraction:
\begin{equation}
s(x) = n_1 \cdot \overline{P_1 A} + n_2 \cdot \overline{A P_2} = n_1 \cdot \sqrt{a^2+x^2} + n_2 \cdot \sqrt{b^2+(d-x)^2} \stackrel{!}= \mbox{Minimum}
\end{equation}
\begin{equation}
\frac{ds}{dx} = n_1 \cdot \frac{x}{\underbrace{\sqrt{a^2+x^2}}_{\sin \alpha_1}}- n_2 \cdot \frac{d-x}{\underbrace{\sqrt{b^2+(d-x)^2}}_{\sin \alpha_2}} \stackrel{!}= 0
\end{equation}
\begin{equation}
\Rightarrow \frac{\sin \alpha_1}{\sin \alpha_2}=\frac{n_2}{n_1}
\end{equation}
\\ 
\textbf{Task 5}\\
images missing\\
\\
\textbf{Task Chapter 11.2.3}
\begin{equation}
f = \frac{a^2-e^2}{4a} = \frac{600^2-255.3^2}{4\cdot 600}mm = 122.84mm
\end{equation}
\begin{equation}
f_x = \frac{f \cdot (f_s-d)}{f_s-f} = \frac{122.84\cdot (80-10)}{80-122.84}mm = -200.71mm
\end{equation}

\subsubsection{Lens equation}
The commonly known lens formula is
\begin{equation}
\frac{1}{f} = \frac{1}{g} + \frac{1}{b}
\end{equation}
where $f$ is the focal distance, $g$ is the object distance and $b$ is the image distance.

\subsubsection{Lense pair}
The lens equation for a pair of lenses $L_1$ and $L_2$ looks a bit different:
\begin{equation}
\frac{1}{f}= \frac{1}{f_1} + \frac{1}{f_2} - \frac{d}{f_1 \cdot f_2}
\end{equation}
where $f_i$ is the focal distance of lense $L_i$ and $d$ is the distance between $L_1$ and $L_2$

\subsubsection{Bessel's method}
We use the method proposed by Bessel to find the focal distance $f$ of a lense by experimental means. Let $a \doteq g + b$ be a fixed quantity. Then there exists for a given focal distance $f < \frac{a}{4}$ only two possible lens positions which produce a sharp image $B$ of object $G$. 

\begin{figure}[H]
	\centering
  \includegraphics[width=0.9\textwidth]{diag/lens_skript.pdf}
	\caption{Finding focal distance using Bessel's method}
	\label{fig:method}
\end{figure}

These lens positions are given by
\begin{equation}
b_{1,2} = \frac{a}{2} \pm \sqrt{\left(\frac{a}{2}\right)^2-a\cdot f}
\end{equation}

\begin{equation}
g_{1,2} = a-b_{1,2}=b_{2,1} = \frac{a}{2} \mp \sqrt{\left(\frac{a}{2}\right)^2-a\cdot f} 
\end{equation}

\begin{equation}
b_1 - b_2 = 2 \cdot \sqrt{\left(\frac{a}{2}\right)^2-a\cdot f}
\end{equation}

Using $e \doteq b_1 - b_2$ and solving for $f$ yields

\begin{equation}
\boxed{f = \frac{a^2-e^2}{4\cdot a}}
\end{equation}

\paragraph{Diverging lenses}
The image of a diverging lens ($f_z < 0$) is virtual and thusly cannot be projected onto a screen. To measure the focal distance $f_z$ we just use a second lens with known focal distance $f_s$ and we get
\begin{equation}
\frac{1}{f} = \frac{1}{f_s} + \frac{1}{f_z} - \frac{d}{f_s \cdot f_z}
\end{equation}
and hence
\begin{equation}
f_z = \frac{f \cdot \left( f_s - d\right)}{f_s - f}
\end{equation}

\paragraph{Lenses with large focal distances}
It is not possible to directly measure focal distances $f > \frac{a}{4}$. Analogous to diverging lenses we just use a second lens with known focal distance $f_s$ to calculate the large focal distance $f_x$

\begin{equation}
f_x = \frac{f \cdot \left( f_s - d\right)}{f_s - f}
\end{equation}

\section{Experiment setup and execution}

\subsection{Used materials}
The materials used in this experiment are the following:
\begin{itemize}
\item a projector which projects a static image
\item a milk glass screen
\item a rail with imprinted scale
\item 4 different lenses, labelled 1C to 4C
\item Some mounts for attaching all of the above to the rail
\end{itemize}
\subsection{Assembly}
For our measurements, the screen is attached at one end of the rail and the projector at the other, facing the screen. In between, a lense is placed and its position varied until two points are found where a sharp image appears on the screen. If we found no positions, we attached a second lens to the first lens and repeated the measurement. This whole process is repeated 5 times for each lens.

\section{Measurements}

\begin{figure}[H]
	\centering
  \includegraphics[width=0.4\textwidth]{diag/general_measurements.pdf}
	\caption{General measurements for our assembly}
\end{figure}
\begin{figure}[H]
	\centering
  \includegraphics[width=0.4\textwidth]{diag/one_lens.pdf}
	\caption{measurements for one mounted lens}
\end{figure}
\begin{figure}[H]
	\centering
  \includegraphics[width=0.4\textwidth]{diag/two_lenses.pdf}
	\caption{measurements for two mounted lenses}
\end{figure}

\section{Analysis and Discussion}

\section{Conclusion}

\begin{thebibliography}{9}

\bibitem{physcript13}
  Peter Wurz,
  \emph{Anleitung zum Physikpraktikum}
  FS2013

\end{thebibliography}

\end{document}
