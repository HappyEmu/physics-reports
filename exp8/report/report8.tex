\documentclass{scrreprt}
\usepackage[english]{babel}
\usepackage[T1]{fontenc}
\usepackage{lmodern}
\usepackage{blindtext}
\usepackage[utf8]{inputenc}
\usepackage{siunitx} %For unit handling%
\renewcommand{\familydefault}{\sfdefault}
\newcommand{\unit}[1]{\ensuremath{\, \mathrm{#1}}}
\usepackage{amssymb, amsmath, cancel, ulem, graphicx, float, tabularx, multirow, bm}
\usepackage{amsmath}
\usepackage{caption}
\usepackage{subcaption}
\usepackage{mathtools}
\usepackage{tikz}
\newcommand*\circled[1]{\tikz[baseline=(char.base)]{
            \node[shape=circle,draw,inner sep=1pt] (char) {#1};}}
\renewcommand{\phi}{\varphi}


\setcounter{secnumdepth}{5}
\setcounter{tocdepth}{5}

\author{Urs Gerber\\09-921-156 \and Gian-Luca Mateo\\11-113-545}
\date{18th of April 2013}

\title{Coupled pendulums}
\subtitle{Practical course report}

\begin{document}

\maketitle

\tableofcontents
\newpage

\chapter{Experiment: Coupled pendulums}

\section{Introduction}

\subsection{Goal of the experiment}
The goal of this experiment is to analyze the visible light spectrum of a mercury vapor lamp. To do this a spectrometer is first calibrated and then used to find the angles of refraction along the edge of a prism made of glass for a certain wavelength. In addition, we examine some physical properties of the prism.
 
\subsection{Theory}
A spectrometer is a device which is used to examine the spectral composition of light 

\section{Experiment setup and execution}

\subsection{Used materials}
The materials used in this experiment are the following:
\begin{itemize}
\item a turntable, with an adjustable platform for the prism
\item a prism
\item a mercury vapor lamp
\item a collimator, mounted to the turntable, turnable 
\item a telescope, mounted to the turntable, turnable
\item an adjustable slit, mountable to the telescope
\end{itemize}

\subsection{Assembly and Execution}



In the first measurement, we first adjusted the platform so the sides of the prism were exactly upright. Then, rotating the table to a position where one side faced the telescope, measuring the angle of the telescope and repeating the procedure for a second side, we measured the angle of the prism.
\\\\
For the second measurement, we lit up the vapor lamp and, rotating the platform, looked for the point where the spectral lines were minimally deflected. We measured that angle for each of the seven spectral lines stated in \cite[p. 159]{physcript13}. 
\\\\
For the third measurements, we mounted the adjustable slit on the telescope and closed it to the narrowest possible position where the yellow lines were still distinguishable. 
\section{Measurements}


\section{Analysis and Discussion}

\section{Conclusion}

\begin{thebibliography}{9}

\bibitem{physcript13}
  Peter Wurz,
  \emph{Anleitung zum Physikpraktikum}
  FS2013

\end{thebibliography}

\end{document}
