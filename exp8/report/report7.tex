\documentclass{scrreprt}
\usepackage[english]{babel}
\usepackage[T1]{fontenc}
\usepackage{lmodern}
\usepackage{blindtext}
\usepackage[utf8]{inputenc}
\usepackage{siunitx} %For unit handling%
\renewcommand{\familydefault}{\sfdefault}
\newcommand{\unit}[1]{\ensuremath{\, \mathrm{#1}}}
\usepackage{amssymb, amsmath, cancel, ulem, graphicx, float, tabularx, multirow, bm}
\usepackage{amsmath}
\usepackage{caption}
\usepackage{subcaption}
\usepackage{mathtools}
\usepackage{tikz}
\newcommand*\circled[1]{\tikz[baseline=(char.base)]{
            \node[shape=circle,draw,inner sep=1pt] (char) {#1};}}
\renewcommand{\phi}{\varphi}


\setcounter{secnumdepth}{5}
\setcounter{tocdepth}{5}

\author{Urs Gerber\\09-921-156 \and Gian-Luca Mateo\\11-113-545}
\date{18th of April 2013}

\title{Coupled pendulums}
\subtitle{Practical course report}

\begin{document}

\maketitle

\tableofcontents
\newpage

\chapter{Experiment: Coupled pendulums}

\section{Introduction}

\subsection{Goal of the experiment}

\subsection{Theory}

\section{Experiment setup and execution}

\subsection{Used materials}
The materials used in this experiment are the following:
\begin{itemize}
\item An assembly (labelled No.$3$) with two ball bearing mounted shafts ($m_s = 21.37\unit{g}$ each), which have threaded bottoms
\item Two weights ($m_w = 214.15 \unit{g}$ each) with threaded holes which can be screwed onto the bottom of the shafts and secured with two nuts ($m_n = 2\unit{g}$ each)
\item two brackets, mountable to the shafts
\item a spring, attachable to the brackets 
\item a ruler, metric scale, $38 \unit{cm}$ long
\item a mechanical stopwatch, $0.2 \unit{s}$ steps
\end{itemize}

\subsection{Assembly}
For our measurements, the materials are assembled as shown in image \ref{fig:assembly}.


\section{Measurements}


\section{Analysis and Discussion}

\section{Conclusion}

\begin{thebibliography}{9}

\bibitem{physcript13}
  Peter Wurz,
  \emph{Anleitung zum Physikpraktikum}
  FS2013

\end{thebibliography}

\end{document}
